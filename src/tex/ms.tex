% Define document class
\documentclass[twocolumn]{aastex631}

% Add here any additional package you need
% \usepackage{foo}

% If you can, leave this package for last
\usepackage{showyourwork}

\newcommand{\dif}{\mathrm{d}}

% Begin!
\begin{document}

% Title
\title{Memory in the Population of GW Signals Detected by LVK}

% Author list
\author[0000-0003-1540-8562]{Will M. Farr}
\email{wfarr@flatironinstitute.org}
\email{will.farr@stonybrook.edu}
\affiliation{Center for Computational Astrophysics, Flatiron Institute, 162 5th Avenue, New York NY 10010, USA}
\affiliation{Department of Physics and Astronomy, Stony Brook University, Stony Brook NY 11794, USA}

\author{and friends}

\begin{abstract}
    Waveform inception.
\end{abstract}

% Main body with filler text
\section{Methodology}
\label{sec:methodology}

Gravitational wave memory is an effect predicted by general relativity, in which
the passage of a gravitational wave causes a permanent displacement in the
positions of test masses. This effect is expected to be present in the signals
detected by the LIGO-Virgo-KAGRA (LVK) collaboration, and its detection would
provide a new test of general relativity and insights into the nature of
gravity.  Something, something, something IR triangle.  It would be wicked cool
to detect, and we may have already observed it and just not know.

Gravitational memory is a non-linear effect in the wave zone.  Suppose a
gravitational wave with strain $h$ is passing through a detector.  The memory
effect causes additional strain, $h_m$, to occur, which is roughly 
\begin{equation}
h_m(t) \sim \frac{1}{2} \int_{-\infty}^t \dif s \, \dot{h}^2(s).
\end{equation}
Most waveform models do not incorporate the memory effect, but it can be
calculated straightforwardly from the strain output by any given waveform model.

Since existing waveform models do not include the memory effect, the catalogs of
events detected by the LVK collaboration do not include it in their estimation
of each observed event's parameters, either.  Suppose we have a set of parameter
samples, taken from the posterior distribution of parameters, $\theta$, for a
given event, $\left\{ \theta_i \mid i = 1, \ldots, N \right\}$ (here the
parameters included in $\theta$ are the usual source parameters---masses, spins,
distances, etc---as well as any other parameters needed to describe the
observation, such as detector calibration parameters, etc).  The log-likelihood
used to produce these samples takes the form 
\begin{multline}
    \log \mathcal{L} = -\frac{1}{2} \left\langle d - R(\theta) h(\theta) \mid d - R(\theta) h(\theta) \right\rangle + C \\ = -\frac{1}{2} \left\langle d - \tilde{h}(\theta) \mid d - \tilde{h}(\theta) \right\rangle + C,
\end{multline}
where $d$ is the data, $h(\theta)$ is the waveform model evaluated at parameters
$\theta$, $R(\theta)$ is the detector response (including calibration effects),
$\tilde{h} = R h$ is the waveform projected into the data using the detector
response, $\langle \cdot \mid \cdot \rangle$ is the usual noise-weighted inner
product, and $C$ is an (unimportant) parameter-independent constant.  Let
$r(\theta)$ denote the \emph{residuals} obtained by evaluating the waveform
model at parameters $\theta$ and subtracting it from the data, i.e. $r(\theta) =
d - R(\theta) h(\theta) = d - \tilde{h}(\theta)$; then the likelihood is 
\begin{equation}
    \log \mathcal{L} = -\frac{1}{2} \left\langle r(\theta) \mid r(\theta) \right\rangle + C.
\end{equation}

If we now wish to add in the effects of memory, we can calculate the memory contribution to the strain, $h_m(\theta)$, and add it to the waveform model, so that the likelihood becomes
\begin{equation}
    \log \mathcal{L} = -\frac{1}{2} \left\langle r(\theta) - \tilde{h}_m(\theta) \mid r(\theta) - \tilde{h}_m(\theta) \right\rangle + C.
\end{equation}
If we wish to \emph{detect} the memory effect, then we can add a ``memory amplitude'' parameter, $A_m$, which multiplies the memory contribution to the waveform, so that the likelihood becomes
\begin{multline}
    \label{eq:likelihood-with-Am}
    \log \mathcal{L} =  \\ -\frac{1}{2} \left\langle r(\theta) - A_m \tilde{h}_m(\theta) \mid r(\theta) - A_m \tilde{h}_m(\theta) \right\rangle \\ + C.
\end{multline}
The expected memory effect in GR is recovered when $A_m = 1$; computing the posterior on $A_m$ allows us to test the consistency of the observed memory effect with the predictions of GR, and to determine whether the data prefer a non-zero memory effect at all.  Conveniently, Eq.~\eqref{eq:likelihood-with-Am} takes the form of a Gaussian on $A_m$, peaking at 
\begin{equation}
    \hat{A}_m = \frac{\left\langle \tilde{h}_m(\theta) \mid r(\theta) \right\rangle}{\left\langle \tilde{h}_m(\theta) \mid \tilde{h}_m(\theta) \right\rangle},
\end{equation}
with standard deviation of 
\begin{equation}
    \sigma_{A_m}^2 = \frac{1}{\left\langle \tilde{h}_m(\theta) \mid \tilde{h}_m(\theta) \right\rangle}.
\end{equation}

Marginalizing $A_m$ out of the likelihood with an (improper) flat prior, we
obtain the marginal likelihood $\bar{\mathcal{L}}$:
\begin{multline}
    \log \bar{\mathcal{L}} = \log \int \dif A_m \exp\left( \log \mathcal{L} \right) = \\ -\frac{1}{2} \left( \left\langle r(\theta) \mid r(\theta) \right\rangle - \hat{A}_m \left\langle r(\theta) \mid \tilde{h}_m(\theta) \right\rangle \right) \\ - \frac{1}{2}\log\left( 2 \pi \left\langle \tilde{h}_m \mid \tilde{h}_m \right\rangle \right) + C.
\end{multline}
If we are given a set of samples as above, drawn from a posterior based on the original, no-memory likelihood $\mathcal{L}$, then the ratio of the marginal likelihood to the original likelihood serves as an importance weight that can be applied to the samples to obtain a new set of samples drawn from the posterior based on the marginal likelihood $\bar{\mathcal{L}}$:
\begin{multline}
    \log w_i = \log \bar{\mathcal{L}}(\theta_i) - \log \mathcal{L}(\theta_i) = \\ \frac{1}{2} \hat{A}_m \left\langle r(\theta_i) \mid \tilde{h}_m(\theta_i) \right\rangle \\ - \frac{1}{2}\log\left( 2 \pi \left\langle \tilde{h}_m \mid \tilde{h}_m \right\rangle \right).
\end{multline}
If the memory effect is not very strong in any individual event, then the importance weights will be close to 1, and the re-weighted samples will be very similar to the original samples.  Each sample can also be augmented by a value of $A_m$ drawn from the Gaussian distribution described above, to obtain a new set of samples drawn from the posterior on $A_m$ as well.

We can see that the key parts of the calculation of the memory posterior over a catalog of events are 
\begin{enumerate}
    \item the calculation of the memory contribution to the waveform, $h_m(\theta)$, for each event;
    \item the computation of the detector response, $R(\theta)$;
    \item and the calculation of the inner products $\left\langle r(\theta_i)
    \mid \tilde{h}_m(\theta_i) \right\rangle$ and $\left\langle
    \tilde{h}_m(\theta_i) \mid \tilde{h}_m(\theta_i) \right\rangle$ using the
    noise-weighted inner product.
\end{enumerate}
The latter two steps should be straightforward to extract from Bilby using the
configuration settings for each original posterior estimation from the catalog.
Keefe can supply the first component.

Once posteriors on $A_m$ are obtained for each event, we can combine them in the
usual hierarchical way (known to Max and Will) to obtain a posterior on the
population distribution of $A_m$ across the catalog, and to determine whether
the data prefer a non-zero memory effect in the population as a whole.

\software
\showyourwork \citep{Luger2021}.

\bibliography{bib}

\end{document}
